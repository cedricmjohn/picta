\documentclass[12pt]{article}
\usepackage[a4paper, total={6in, 8in}, margin=1in]{geometry}
\usepackage{biblatex}
\usepackage{lmodern}
\usepackage{parskip}
\addbibresource{References.bib}

\begin{document}
	%==================================
	%	TITLE PAGE
	%==================================
\begin{titlepage}
	\begin{center}
		\vspace*{1cm}
		
		Imperial College London \\
		Department of Earth Science and Engineering \\
		MSc in Applied Computational Science and Engineering
		
		\vspace{1cm}
		
		Independent Research Project \\
		Project Plan
		
		
		\vspace{1cm}
		
		
		{\fontsize{18}{104}\selectfont \textbf{ CoNuS-Viz: A Visualization Package for the Concurrent Library for Numerical Simulations}}

		
		\vspace{1cm}
		
		by \\ 
		Fazal Khan
		
		\vspace{1.5cm}
		
		fk4517@ic.ac.uk \\
		
		GitHub login: acse-fk4517
		
		
		\vfill
		
		Supervisor: \\
		Professor Cédric John
		
		\vspace{2.0cm}
		
		June 2020
		
	\end{center}
\end{titlepage}	
%==================================
%	Introduction
%==================================
\section{Introduction}

\subsection{Stratigraphic Forward Modelling}

Many real-world scientific processes may be formulated as a model that evolves with time. Formulating a process this way allows it to be modelled using computational techniques. Suppose $\vec y$ are the variables of interest stacked in a vector, then a forward model describing their evolution with respect to time may be written as:

\begin{center}
	\vspace{0.5cm}
	$\vec y_{t+1} = f(t, \vec y_t)$
	\vspace{0.5cm}
\end{center}


This technique has widespread applicability in many different scientific fields. One particular field where forward modelling has been especially useful is in Stratigraphy. Stratigraphy is concerned with the processes which lead to the organization of rocks in space and time.
Stratigraphic Forward Models (SFM) are forward models which have been used to simulate realistic stratigraphic attributes based on a set of reasonable geologic parameter values, such as tectonic movement and erosion \cite{10.2110/pec.99.62.0069}.

When studying observational data, it is often important to understand what set of initial geological conditions may have caused the stratigraphic patterns in the observations. Problems of this type can be generally classified as inverse problems \cite{10.1260/0144598011492363}. SFMs can be used to generate a set of outputs which in turn are compared to observations. Outputs are generated systematically for a range of initial parameter values, and those outputs which are close to the observations within a pre-defined bound are then taken as solutions to this inverse problem. Used in this way SFMs are an integral part of solving inversion problems in stratigraphy.

SFMs are computationally intensive methods which may require many outputs to be generated to find a satisfactory solution. In addition to generating many outputs, the set of geological parameters, $\vec y_i$ is often very large too \cite{10.1260/0144598011492363}. For these reasons computational techniques that can deal with large models efficiently is very important. One way to address this is to use concurrency and parallelism when performing calculations on the underlying data. Such techniques have been used in scientific modelling for increasing performance and reducing computation time. Domains ranging from bio-physical modelling to computational fluid dynamics have benefited from technology such as GPUs that allows these techniques to be more readily exploited \cite{4490127}.

\subsection{The Concurrent Library for Numerical Simulations (CoNuS)}
CoNuS is an experimental open-source library developed by Professor Cédric John from the Carbonate Research Group at Imperial College London. It aims to be a general forward modelling library that leverages the Scala programming language and functional programming techniques. The overall aim of the library is to abstract over the mechanics of forward model computations and allow users from any discipline to easily compose models using their domain knowledge and expertise. As it currently stands the library lacks any way to visualize the output it generates directly.

Scala is a statically typed, compiled language that targets the Java Virtual Machine (JVM) \cite{scala}. Scala combines both object-orientated programming and functional programming concepts to create highly expressive programs that can support many levels of abstraction over the underlying computation. A result of this expressiveness is the Scala actor model which allows concurrent code to be written ergonomically, in a type-safe manner. Originally the actor model was implemented to deal with the JVMs shared-memory thread model, which suffered from high memory consumption and context switching costs \cite{haller_odersky_2009}. Making use of functional programming techniques available in Scala, such as partial functions, the original implementation was able to run on unmodified JVMs, requiring no special compiler support. The Scala actors library has since been replaced by the Akka library, which is built on the same concepts \cite{}.



\subsection{The CoNuS Forward Modelling Library}


Describe your objectives and/or hypotheses, and outline the tasks completed during the independent research project. 

Describe state-of-the-art of solutions to the problem, including commercial and academic approaches, and cite these using the reference style described in the “Guide for Authors” document from the SoftwareX journal.

Describe briefly the requirements of your solution (Software Requirement Specification –SRS). 

Clearly state how your independent research project goes beyond the state-of-the-art and what original work you have done.

	
This is a test. \cite{ALSALMI201945}

\section{References} 
\printbibliography[heading=none]

\end{document}